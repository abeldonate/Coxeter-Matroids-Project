\documentclass{myclass}
\usepackage{verbatim}
\usepackage{array}
\usepackage{listings}
\usepackage{fancyvrb}
\usepackage{enumitem}

\usepackage[utf8]{inputenc}
\usepackage[T1]{fontenc}
\usepackage{textcomp}
\usepackage{multicol}
\usepackage{mathtools}
\usepackage{amsmath}
\usepackage{wrapfig}
\usepackage{amssymb}
\usepackage{amsmath,amsfonts,amssymb,amsthm,epsfig,epstopdf,titling,url,array}
\usepackage{hyperref}
\usepackage{eso-pic}
\usepackage{pgf}
\usepackage{tikz}
\usepackage{graphicx}

% figure support
\usepackage{import}
\usepackage{xifthen}
\pdfminorversion=7
\usepackage{pdfpages}
\usepackage{transparent}
\usepackage{xcolor}

\setlength{\parindent}{0em}
\setlength{\parskip}{1em}

\newtheorem*{definition}{Definition}
\newtheorem*{theorem}{Theorem}
\newtheorem*{proposition}{Proposition}

\newcommand{\incfig}[1]{%
\center
\def\svgwidth{0.9\columnwidth}
\import{./figures/}{#1.pdf_tex}
}

\newcommand{\incsvg}[1]{%
\center
\def\svgwidth{0.9\columnwidth}
\import{./figures/}{#1.svg}
}

\newcommand{\incimg}[1]{%
\center
\includegraphics[width=0.9\columnwidth]{images/#1}
}
\pdfsuppresswarningpagegroup=1

\title{Notes on Coxeter Matroids}

\begin{document}
\maketitle
\tableofcontents
\newpage

\section{Matroids}
\begin{definition}[Matroid]
A base of a matroid M over a given a ground set $[n]$ is $\mathcal{B}(M)\subseteq \choose{[n], r}$, where $r$ is the rank of the matroid. The set  $\mathcal{B}$ must fulfill:
\begin{itemize}[topsep=-6pt, itemsep=0pt]
  \item $A, B \in \mathcal{B}, a\in A-B \Rightarrow \ \exists b\in B-A : (A-\{a\})\cup \{b\}\in \mathcal{B}$
\end{itemize}
\end{definition}

\section{Permutahedron}
\subsection{Regular permutahedron}
The permutahedron $\Pi_n$ is generated by the convex hull of the vertices $V = \{(\sigma(1), \ldots \sigma (n)) : \sigma \in S_n \}$

There is a (fancy) bijection between the flags of $[n]$ and the faces of permutahedron $\Pi_n$ as shown in the picture.

Flags could be interpreted as ordered partitions. One example of the three points of view as follows: $F = \{\{3\}, \{1, 2, 3, 4\}\} \iff 3|124 \iff$ "the face whose vertices have a $3$ in the first position and the other three are free permutations".

\begin{minipage}{\textwidth}
\incfig{PermutahedronBijectionFlags}
\end{minipage}

\subsection{Generalized permutahedra}

\begin{definition}[Hypersimplex] 
  $\Delta(n, k)=\{(x_1, \ldots, x_n): x_1 + \ldots+ x_n = k\}$
\end{definition}

The basis of $\Delta(n,k)$ (vertices of the polytope) is formed by vectors with $k$ ones and  $n-k$ zeroes.

\begin{definition}[Generalized Permutahedron] Convex polytope with all the edges parallel to $e_i-e_j$
\end{definition}

Permutahedron vertices came from a subset of the vertices of $\Delta(n, k)$

 \begin{definition}[Matroid polytope] Matroid generated by the permutahedron whose vertices are a subset of $\Delta(n,k)$



\end{definition}





\section{Tropical geometry}
The idea behind tropical geometry is 

\end{document}

























