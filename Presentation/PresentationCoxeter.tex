%%%%%%%%%%%%%%%%%%%%%%%%%%%%%%%%%%%%%%%%%
% Beamer Presentation
% LaTeX Template
% Version 1.0 (10/11/12)
%
% This template has been downloaded from:
% http://www.LaTeXTemplates.com
%
% License:
% CC BY-NC-SA 3.0 (http://creativecommons.org/licenses/by-nc-sa/3.0/)
%
%%%%%%%%%%%%%%%%%%%%%%%%%%%%%%%%%%%%%%%%%

%----------------------------------------------------------------------------------------
%	PACKAGES AND THEMES
%----------------------------------------------------------------------------------------

\documentclass{beamer}


\newcommand{\incfig}[1]{%
\center
\def\svgwidth{0.9\columnwidth}
\import{./figures/}{#1.pdf_tex}
}


\newcommand{\incimg}[1]{%
\center
\includegraphics[width=0.9\columnwidth]{images/#1}
}


\renewcommand{\v}{\vspace{1em}}

\mode<presentation> {

% The Beamer class comes with a number of default slide themes
% which change the colors and layouts of slides. Below this is a list
% of all the themes, uncomment each in turn to see what they look like.

%\usetheme{default}
%\usetheme{AnnArbor}
%\usetheme{Antibes}
%\usetheme{Bergen}
%\usetheme{Berkeley}
%\usetheme{Berlin}
%\usetheme{Boadilla}
%\usetheme{CambridgeUS}
%\usetheme{Copenhagen}
%\usetheme{Darmstadt}
%\usetheme{Dresden}
%\usetheme{Frankfurt}
%\usetheme{Goettingen}
%\usetheme{Hannover}
%\usetheme{Ilmenau}
%\usetheme{JuanLesPins}
%\usetheme{Luebeck}
\usetheme{Madrid}
%\usetheme{Malmoe}
%\usetheme{Marburg}
%\usetheme{Montpellier}
%\usetheme{PaloAlto}
%\usetheme{Pittsburgh}
%\usetheme{Rochester}
%\usetheme{Singapore}
%\usetheme{Szeged}
%\usetheme{Warsaw}

% As well as themes, the Beamer class has a number of color themes
% for any slide theme. Uncomment each of these in turn to see how it
% changes the colors of your current slide theme.

%\usecolortheme{albatross}
%\usecolortheme{beaver}
%\usecolortheme{beetle}
%\usecolortheme{crane}
%\usecolortheme{dolphin}
%\usecolortheme{dove}
%\usecolortheme{fly}
%\usecolortheme{lily}
%\usecolortheme{orchid}
%\usecolortheme{rose}
%\usecolortheme{seagull}
%\usecolortheme{seahorse}
%\usecolortheme{whale}
%\usecolortheme{wolverine}

%\setbeamertemplate{footline} % To remove the footer line in all slides uncomment this line
%\setbeamertemplate{footline}[page number] % To replace the footer line in all slides with a simple slide count uncomment this line

%\setbeamertemplate{navigation symbols}{} % To remove the navigation symbols from the bottom of all slides uncomment this line
}

\usepackage{graphicx} % Allows including images
\usepackage{booktabs} % Allows the use of \toprule, \midrule and \bottomrule in tables


\usepackage{verbatim}
\usepackage{array}
\usepackage{listings}
\usepackage{fancyvrb}
\usepackage{enumitem}

\usepackage[utf8]{inputenc}
\usepackage[T1]{fontenc}
\usepackage{textcomp}
\usepackage{multicol}
\usepackage{mathtools}
\usepackage{amsmath}
\usepackage{wrapfig}
\usepackage{amssymb}
\usepackage{amsmath,amsfonts,amssymb,amsthm,epsfig,epstopdf,titling,url,array}
\usepackage{hyperref}
\usepackage{eso-pic}
\usepackage{pgf}
\usepackage{tikz}
\usepackage{graphicx}
\usepackage{tikz-cd}

\usepackage{import}
\usepackage{xifthen}
\usepackage{pdfpages}
\usepackage{transparent}
\usepackage{xcolor}

\setlength{\parindent}{0em}
\setlength{\parskip}{1em}



%\pdfsuppresswarningpagegroup=1
%----------------------------------------------------------------------------------------
%	TITLE PAGE
%----------------------------------------------------------------------------------------

\title{Coxeter Matroids} % The short title appears at the bottom of every slide, the full title is only on the title page

\author{Abel Doñate Muñoz} % Your name
\institute[UPC] % Your institution as it will appear on the bottom of every slide, may be shorthand to save space
{
Universitat Politècnica de Catalunya \\ % Your institution for the title page
\medskip
\textit{abel.donate.munoz@gmail.com} % Your email address
}
\date{\today} % Date, can be changed to a custom date

\begin{document}

\begin{frame}
\titlepage % Print the title page as the first slide
\end{frame}

\begin{frame}
\frametitle{Overview} % Table of contents slide, comment this block out to remove it
\tableofcontents % Throughout your presentation, if you choose to use \section{} and \subsection{} commands, these will automatically be printed on this slide as an overview of your presentation
\end{frame}

%----------------------------------------------------------------------------------------
%	PRESENTATION SLIDES
%----------------------------------------------------------------------------------------

%------------------------------------------------
\section{What is a matroid?} % Sections can be created in order to organize your presentation into discrete blocks, all sections and subsections are automatically printed in the table of contents as an overview of the talk
%------------------------------------------------

\subsection{Abstract definition}

\begin{frame}
\frametitle{What is a matroid?}
\begin{block}{Abstract definition (Basis)}
  Given a ground set $[n]$, a subset $\mathcal{B}(M)\subseteq [n]$ is the set of basis of a matroid $M$ if satisfies the exchange axiom:
   \[
  \ \forall A, B \in \mathcal{B}(M), a \in A-B \ \exists b\in B-A \ : \ A-\{a\}\cup {b} \in \mathcal{B}(M)
  \] 
\end{block}
\begin{block}{Rank}
It can be seen that the cardinal of the elements of $\mathcal{B}(M)$ must be the same. We will call it $k = rank(M)$
\end{block}

The elements of  $\mathcal{B}(M)$ are called \textit{Maximal independent sets}. We will notice the intuition of this naming in later slides.
\end{frame}

\subsection{Representable Matroids}
\begin{frame}
\frametitle{Representable Matroids}
Matroids can be represented in several ways. The following diagram shows the different types of matroids and its relations.
\begin{columns}[c]
\column{0.7\textwidth}
\begin{block}{Diagram of matroids}
\incfig{VennMatroids}
\end{block}
\end{columns}
\end{frame}

\begin{frame}{Representable Matroids}
\begin{block}{Definition}
  In a representable matroid we assign a vector to every element of the ground set $i\in [n] \to  v_i \in V$. Now all the maximal independent sets of the matroids can be thought as sets of vectors that are a basis of a vector subspace $W\subseteq V$ of rank $k$.
\end{block} 
\vspace{-1em}
\begin{columns}[c]
\column{0.4\textwidth}
\begin{block}{Example}
\incfig{RepresentableMatorid}
\end{block}
\column{0.5\textwidth}
\begin{block}{Independent sets}
We notice that $v_1, v_2, v_3$ lie all in the same plane, so they are linearly dependent.
We can make a basis of $\mathbb{R}^3$ in the following ways
\[
\langle v_1, v_2, v_4 \rangle , 
\langle v_1, v_3, v_4 \rangle ,
\langle v_2, v_3, v_4 \rangle
\] 
So the basis of the matroid are
\[
  124, 134, 234
\] 
\end{block}
\end{columns}
\end{frame}


%#########################################
%#########################################

\section{Coxeter Groups}
\subsection{Definition}
\begin{frame}{Coxeter Groups}
  \begin{block}{Definition}
    Given a set of generators $S = \{s_i\}$, a Coxeter group is a group whose presentation $\langle s_1, s_2, \ldots, s_n  \rangle $ satisfy
\begin{itemize}[topsep=-6pt, itemsep=0pt]
  \item $(s_is_j)^{m_{ij}} = 1$
  \item $m_{ii} = 1$
  \item $m_{ij}\ge 2 \ \forall i\neq j$
\end{itemize}
  \end{block}
\begin{columns}[c]
\column{0.65\textwidth}
\begin{block}{Hyperplanes}
Coxeter groups are usually thought as finite groups of reflections into a vector space $V$. Each generator $s_i$ is associated with a hyperplane $\rho_i$.

Then, if we consider a vector in the vector space $v$, we can define the action of each element $s_i$ of the group over $v$ as the reflections by $\rho _i$. $s_iv = v - 2\frac{\langle \rho_i, v \rangle }{\langle \rho _i, \rho _i \rangle}\rho _i$
\end{block}
\column{0.3\textwidth}
\begin{block}{Example}
\incfig{CoxeterHyperplanes}
\end{block}
\end{columns}
\end{frame}

\subsection{Dynkin diagrams}
\begin{frame}{Dynkin diagrams}
\begin{block}{Dynkin diagram}
 We can assign every Coxeter group a diagram in the following way:
 \begin{itemize}[topsep=-6pt, itemsep=0pt]
   \item - Nodes are the set of generators $S$
   \item - We connect nodes $s_i, s_j$ by an edge if $m_{ij}\ge 3$
   \item - We label the edges with  $m_{ij}$ if it is $\ge 4$
 \end{itemize}
\end{block}
  \begin{columns}[c]
    \column{0.7\textwidth}
\begin{block}{Examples of Dynkin diagrams} 
  \incfig{TypesCoxeter}
\end{block}
  \end{columns}
\end{frame}

\begin{frame}{Coxeter Groups chart}
  All the finite Coxeter Groups can be classified in the following chart:
 \begin{block}{Chart}
   \incfig{CoxeterChart}
 \end{block} 
\end{frame}

\section{Coxeter Matroids}
\subsection{Definition}
\begin{frame}{Definition of Coxeter Matorid}
  Let $W$ a Coxeter group and  $F$ the set of vectors generated by the actions of  $W$ over a starting vector  $\overline{x}$.

  Let $P$ the polytope generated by the convex hull of the set of vertices $F$.

\begin{columns}[c]
\column{0.4\textwidth}
\begin{block}{Definition}
  A subset of vertices $U\subseteq F$ is a Coxeter Matroid if the convex hull $Q$ generated by $U$ has all the edges parallel to the edges of  $P$.
\end{block} 
\column{0.5\textwidth}
\begin{block}{Example of Coxeter Matroid}
 \incfig{TypeBn} 
\end{block}
\end{columns}
\end{frame}




\subsection{Type $A_n$}

\begin{frame}{Rings}
  We should take into account that there is not a bijection between Dynkin diagrams and Polytopes generated by reflection. A key point in where do we place the first point we are considering the orbit.
  \begin{block}{Rings}
   Considering the base vector, we see if belongs some hyperplane:
   \begin{itemize}[topsep=-6pt, itemsep=0pt]
     \item If $\overline{x}\in \rho _i$ keep the corresponding node untouched on Dynkin diagram
     \item If $\overline{x}\not\in \rho _i$ round the corresponding node with a ring on Dynkin diagram
   \end{itemize}
  \end{block}
  \begin{columns}[c]
  \column{0.55\textwidth}
  \begin{block}{Combinatorically equal polytopes}
  The resulting polytopes differ depending on the distribution of rings. Two polytopes with the same distribution of rings are \textbf{combinatorically} the same (although are not the same polytope).
  \end{block}
  \column{0.35\textwidth}
  \begin{block}{Example}
   \incfig{Cube} 
  \end{block}
  \end{columns}
\end{frame}

\begin{frame}{More examples of rings}
  We observe how the combinatoric class of the polytope changes if the vector $\overline{x}$ belongs to some hyperplane or not
  \begin{block}{Example}
   \incfig{DifferentRings} 
  \end{block}
\end{frame}


\begin{frame}{Type $A_n$}
\begin{block}{Matroid (base) polytope}
A matroid (base) polytope is a geometrical representation of the basis of a matroid in a vector space. 

Given a matroid $M$ of rank $k$ and the set of basis $\mathcal{B}(M)\subseteq \binom{[n]}{k}$, we assign each basis to its indicator vector $A\in \mathcal{B}(M) \to e_A = e_{i_1} + \ldots + e_{i_k}\in V$. Thus, the convex hull of such vectors is a polytope $P\subseteq \Delta(n,k)$ 
\end{block}  
\begin{columns}[c]
\column{0.38\textwidth}
\vspace{-1em}
\begin{block}{Example}
\incfig{ExampleMatroidPolytope}
\end{block}
\column{0.55\textwidth}
\vspace{-1em}
\begin{block}{Matroid polytope}
  \[
  \mathcal{B}(M) = \{12, 13, 14, 23, 24\}
  \] 
  \vspace{-1em}
  \[
  \text{Vertices } \begin{cases}
    12 \to (1, 1, 0, 0)\\
	13 \to (1, 0, 1, 0)\\
	14 \to (1, 0, 0, 1)\\
	23 \to (0, 1, 1, 0)\\
	24 \to  (0, 1, 0, 1)
  \end{cases}
  \] 
\end{block}
\end{columns}
\end{frame}



\begin{frame}{Type $A_n$}
  \begin{block}{Withoff construction}
  The number of rings of the Dynking diagram establishes the numbers of ones of the vectors ($k$). Then, the corresponding polytope will differ depending on the distribution of the rings in Dynkin diagram.
  \end{block}
  \begin{block}{Example}
   \incfig{Withoff} 
  \end{block}
\end{frame}




\begin{frame}{M-convex functions}
Given a set of points $T$, we define a height function as  $h:T \to \mathbb{R}$.

\begin{definition}[Regular subdivision]
A subset $S$ of points is a (lower) regular subdivision induced by $h$ if the convex hull of $S$ is described by the lower convex hull of the polytope $T\times h(T)$ (not well explained)
\end{definition}

\begin{definition}[M-convex function]
The height function $h$ is said to be \textit{M-convex} if the regular subdivision induced by $h$ is permutahedral.
\end{definition}

We will use the notation $Sij := S\cup \{i\}\cup \{j\}$

\end{frame}

\begin{frame}{3-Term Plücker Relations}

\begin{definition}[3-Term Plücker Relations] Let $h $ be a height function on $\Delta(d, n)$.
We say that 3TPR holds if for each $S\in \binom{[n]}{ d-2}$ and $i, j, k, l \not\in S$, the minimum
\[
\min \Big\{h(Sij) + h(Skl), \ h(Sik) + h(Sjl),\  h(Sil) +  h(Sjl)\Big\}
\] 
is attained at least twice.
\end{definition}

\begin{theorem}[] A height function induces a permutahedral regular division if the \textbf{3-Term Plücker Relations} (3TPR) holds.
\end{theorem}
\end{frame}
  

\begin{frame}{Type $B_n$}
 The task is to find a condition similar to 3TPR for type $B_n$. That is, a condition for a height function to form a regular subdivision that partitions the polytope in subpolytopes whose edges are parallel to the original polytope. 
 \begin{block}{Example}
  \incfig{TypeBn} 
 \end{block}
\end{frame}

\begin{frame}{Description of groups $B_n$}
 \begin{block}{Elements of the group}
The group $B_n$ is isomorphic to the group of signed permutations. Then the vertices can be thought as an state of a table with $n$ labeled cards, each of one up or down.
 \end{block} 
 \begin{block}{a}
   We have two types of elements in $S$:
    \begin{itemize}[topsep=-6pt, itemsep=0pt]
     \item $\tau $ corresponds to the change of sign of the first coordinate.
	 \item $s_i$ corresponds to the transposition and change of sign of the coordinates $i$ and  $i+1$
   \end{itemize}
 \end{block}
 \begin{theorem}[Theorem]
A subpolytope $\text{conv}(U) = Q \subseteq P$ is a Coxeter Matroid if and only if all the edges of Q are or the form $p \to  wp $ with $w$ one of the following forms
 \begin{itemize}[topsep=-6pt, itemsep=0pt]
  \item $w = c_i$, where  $c_i$ means change the sign of  $i$ component
  \item  $w = t_{ij}$, where $t_{ij}$ is the transposition of the $i, j$ components
  \item  $w = c_{ij}t_{ij}$. Transposition and change of sign
\end{itemize}
 \end{theorem}
\end{frame}

%------------------------------------------------

\begin{frame}{Bibliografía}
\footnotesize{
\begin{thebibliography}{99} % Beamer does not support BibTeX so references must be inserted manually as below
\bibitem{p1} Michael Joswig, Georg Loho, Dante Luber, Jorge Alberto Olarte
\newblock Generalized Permutahedra and positive flag dressians
\newblock \emph{Arxiv} 2111.13676v2
\end{thebibliography}

\begin{thebibliography}{99} % Beamer does not support BibTeX so references must be inserted manually as below
\bibitem{p2} Alexandre Borovik, I.M. Gelfand, Neil White
\newblock Coxeter Matroids
\newblock ISBN-13:978-1-4612-7400-1
\end{thebibliography}

}
\end{frame}
















\end{document}
